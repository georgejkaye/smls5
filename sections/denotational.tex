\section{Denotational semantics}

\begin{frame}
    \frametitle{We need some meaning}

    \centering

    \huge
    \textbf{What is the meaning?}

\end{frame}

\begin{frame}
    \frametitle{We need some meaning}

    \centering
    \scalebox{4}{\textbf{Denotational}}

    \scalebox{4}{\textbf{semantics}}

\end{frame}

\begin{frame}
    \frametitle{Interpreting the values}

    \await
    Values are interpreted in a \alert{lattice}:

    \await
    \begin{minipage}{0.49\textwidth}
        \[
            \scalebox{1.5}{\tikzfig{circuits/a4}}
        \]
    \end{minipage}
    \await
    \begin{minipage}{0.49\textwidth}
        \begin{align*}
            \dsptikzfig{strings/structure/monoid/init}[colour=comb]
            \, & \mapsto\, \bot         \\
            \dsptikzfig{circuits/components/values/vs}[val=\belnapfalse]
            \, & \mapsto\, \belnapfalse \\
            \dsptikzfig{circuits/components/values/vs}[val=\belnaptrue]
            \, & \mapsto\, \belnaptrue  \\
            \dsptikzfig{circuits/components/values/vs}[val=\top]
            \, & \mapsto\, \top
        \end{align*}
    \end{minipage}
\end{frame}
\begin{frame}
    \frametitle{Let's make everything a function}

    \await
    \setlength{\tabcolsep}{1.5em}
    \renewcommand{\arraystretch}{2}

    \begin{center}
        \begin{tabular}{lrl}
            \dsptikzfig{circuits/components/gates/gate}[gate=g]
             &
            \alert{monotone functions}
             &
            \(\morph{\overline{g}}{\valuetuple{m}}{\values}\)
            \\
            \await
            \hspace{0.175cm}
            \dsptikzfig{strings/structure/monoid/init}[colour=comb]
             &
            \alert{initialise}
             &
            \(() \mapsto (\bot)\)
            \\
            \await
            \dsptikzfig{strings/structure/comonoid/copy}[colour=comb]
             &
            \alert{copy}
             &
            \(x \mapsto (x, x)\)
            \\
            \await
            \dsptikzfig{strings/structure/monoid/merge}[colour=comb]
             &
            \alert{join in the lattice}
             &
            \((x, y) \mapsto x \ljoin y\)
            \\
            \await
            \dsptikzfig{strings/structure/comonoid/discard}[colour=comb]
             &
            \alert{discard}
             &
            \(x \mapsto ()\)
        \end{tabular}
        \await

        \vspace{0.5em}

        Feedback is interpreted as the \alert{least fixed point}.
    \end{center}
\end{frame}
\begin{frame}
    \frametitle{Functions are not enough}

    \centering
    \LARGE
    How do we model \alert{delay}?

    \await
    \alert{Streams!}
\end{frame}
\begin{frame}
    \frametitle{Streams}

    A \alert{stream} \(\stream{\values}\) is an infinite sequence of values.
    \[
        v_0
        \streamcons
        v_1
        \streamcons
        v_2
        \streamcons
        v_3
        \streamcons
        v_4
        \streamcons
        v_5
        \streamcons
        v_6
        \streamcons
        v_7
        \streamcons
        \cdots
    \]

    \await
    A \alert{stream function} \(\stream{\values} \to \stream{\values}\) consumes and
    produces streams.
    \[
        f(
        v_0
        \streamcons
        v_1
        \streamcons
        v_2
        \streamcons
        v_3
        \streamcons
        v_4
        \streamcons
        \cdots
        ) =
        w_0
        \streamcons
        w_1
        \streamcons
        w_2
        \streamcons
        w_3
        \streamcons
        w_4
        \streamcons
        \cdots
    \]
\end{frame}
\begin{frame}
    \frametitle{Interpreting the sequential components}

    \LARGE

    \await
    \[
        \dsptikzfig{circuits/components/values/vs}[val=v]()
        \coloneqq
        v \streamcons \bot \streamcons \bot \streamcons \bot \streamcons \cdots
    \]

    \await
    \vspace{0.5em}

    \[
        \normalsize
        \dsptikzfig{circuits/components/waveforms/delay}
        \LARGE(
        v_0 \streamcons v_1 \streamcons v_2 \streamcons \cdots
        )
        \coloneqq
        \bot \streamcons v_0 \streamcons v_1 \streamcons v_2 \streamcons \cdots
    \]
\end{frame}
\begin{frame}{Maybe there are too many streams}

    \centering
    \LARGE
    Does every circuit correspond to a stream function \(
    \valuetuplestream{m} \to \valuetuplestream{n}
    \)?

    \Huge
    \await
    No.

    \scriptsize
    \await
    (but this is to be expected!)
\end{frame}
\begin{frame}
    \frametitle{Restricting the stream functions}

    \await
    \Large
    Circuits are \alert{causal}.

    \await

    \normalsize
    They can only depend \alert{what they've seen so far}.

    \await

    \Large
    Circuits are \alert{monotone}.

    \await

    \normalsize
    They are constructed from \alert{monotone functions}.

    \await

    \Large
    Circuits are \alert{finitely specified}.

    \normalsize
    Their streams have \alert{finitely many stream derivatives}


\end{frame}
\begin{frame}
    \frametitle{These are the streams we're looking for}

    \Large

    \begin{theorem}
        A stream function is the interpretation of a sequential circuit
        if and only if it is \textbf{causal}, \textbf{monotone} and has
        \textbf{finitely many stream derivatives}.
    \end{theorem}

    \await

    \begin{center}
        \LARGE
        Sound and complete
        \alert{denotational semantics}!
    \end{center}
\end{frame}